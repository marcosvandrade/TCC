\chapter{Introdução}
\label{chp:introduction}
%Faça aqui, uma introdução geral da área do conhecimento à qual o tema escolhido
%está ligado. 
O tema está diretamente ligado a área de desenvolvimento de software e terá como
objetivo o desenvolvimento de uma Aplicação Web e Mobile utilizando a linguagem
de programação JavaScript. No frontend será utilizado o React, no backend será
utilizado o Nodejs e o banco de dados será o PostGresql. O React Native
possibilitará a portabilidade do sistema para mobile.
\section{Tema}
% A melhor forma de determinar o tema abordado é através de hipóteses. A hipótese
% consiste em uma afirmativa que você considera verdadeira e que vai provar ou
% buscar provar ao longo de seu trabalho. Outra forma é delimitando o problema em
% forma de uma pergunta de partida. Apresente uma visão geral do assunto que será
% abordado no trabalho.
O tema escolhido será desenvolver um Sistema de Administração de Condomínios
personalizado. Atualmente, existem sistemas com soluções personalizadas e acessíveis
para administração de condomínios?
\section{Problema}
% Dedique este tópico a esclarecer o que o pretende de fato com o seu esforço de
% pesquisa. Problema é a questão a ser respondida pelo trabalho, que motivou a sua
% realização. É uma questão que já tomou se formou em sua mente, derivada de
% teorias da área pesquisada e de sua observação sobre um fenômeno.  Normalmente
% se utilizam os subitens abaixo como meios de se determinar claramente os
% objetivos, o que também colabora para a delimitação do escopo do trabalho. Está
% estreitamente ligado ao objetivo geral, que, normalmente, consiste em encontrar
% a resposta para o problema de pesquisa.  O que você viu que é um problema que
% precisa de solução? É viável? Você consegue fazer? O problema é sempre uma
% dificuldade, uma lacuna.
Por morar em um condomínio, foi observado que o síndico do bloco não possui
nenhum sistema para administração de condomínio e que mensalmente são encadernados
mais de 100(cem) páginas dentre comprovantes, prestações de conta, notas fiscais,
controle de empregados, etc.

Ao ser realizada uma pesquisa nos condomínios adjacentes foi percebido a mesma
problematica e que para resolver essas dores poderia surgir uma excelente
oportunidade de negócio.

Ao ser realizado a entrevista inicial com o síndico do bloco, obteve-se o
aceite de imediato e o este condomínio será o piloto no desenvolvimento
do TCC, que em troca receberá o sistema com custo zero.

\subsection{Objetivo geral}
% É a resposta ao problema especificado acima, ou seja, aquilo que se pretende
% fazer e que, depois de atingido, estará concluído o trabalho.. Alguns verbos
% utilizados para determinar o objetivo geral: contribuir / facilitar / subsidiar
% / propor / clarear / permitir / agregar / compreender.
Desenvolver uma aplicação Web para administração de condomínios e que
terá como objetivo geral a criação de um portal, onde serão oferecidos
os serviços e que encontra-se hospedado em \url{http://cliquesindico.com.br}.
Este portal será um "cartão de visitas" para os futuros contratos de
desenvolvimento de sistemas personalizados para condomínios.

A idéia principal é modularizar a aplicação através de subdomínios, onde cada 
bloco ou condomínio terá um subdomínio único e com um sistema personalizado
de acordo com a necessidade do condomínio.

Para o condomínio que será piloto neste projeto, foi criado o subdomínio
\url{http://sqn306b.cliquesindico.com.br}, onde na página inicial terá uma página
de login, onde inicialmente será definidos o perfil de administrador e de usuário
comum. Após se logar na aplicação haverá um redirecionamento para a página
principal do condomínio, onde encontrará um mural com avisos gerais do condomínio.

Serão enviados e-mail aos condominos com avisos, notificações, cobranças, etc

Será criado uma página para prestação de contas mensal do condomínio.

Após a conclusão do sistema, o síndico e a diretoria do condomínio terão total
autonomia para inserir novos avisos, atualizar a prestação de contas e imprimir
relatórios.

\subsection{Objetivos específicos}
% Os objetivos específicos detalham os objetivos gerais através de etapas ou fases
% de pesquisa. Devem ser utilizados verbos no infinitivo, assinalando as ações
% propostas para alcançar o objetivo geral. Os verbos utilizados aqui são os de
% ação, que serão utilizados na metodologia.
\begin{itemize}
    \item Realizado a entevista inicial para determinação da dores e levantamento
    inicial de requisitos.
    
    \item Criar o cronograma do Projeto.
    
    \item Criação de um portal onde serão inseridas as informações gerais sobre
negócio, que terá como diferencial a personalização e o preço competitivo.
    
    \item Cada condomínio terá um subdomínio personalizado e idependente do
restante do sistema.

    \item Será criado um módulo para login no sistema do condomínio em questão
que terá a opção de cadastro, caso o usuário ainda não esteja cadastrado,
e o acesso será realizado através e-mail e senha. O sistema notificará o síndico
do condomínio correspondente, a fim de que valide o cadastro ou revogue o acesso
caso o usuário não tenha relação com o condomínio.

    \item Será recolhido junto a administração do condomínio a documentação 
inicial para que o sistema começe a ser desenvolvido, tais como, e-mail
dos moradores, avisos iniciais para inserção na página principal e modelo de
ajuste de contas.

    \item Desenvolvimento do sistema.
    
    Desenvolver o frontend, backend, api e banco de dados. Para isso será 
utilizado a linguagem JavaScript, React no frontend, NodeJS para APIs e para
o backend e Postgreesql para o banco de dados.

    Para modelagem do banco será utilizado o BrModelo e para criação do banco o PGAdmin.

    \item Realização dos testes para validação inicial do sistema, junto a administração
do condomínio e levantamento de problemas e melhoramentos.

    \item Documentar o sistema
    
\end{itemize}





Implementar os diferentes perfis de acesso, de acordo com a real necessidade
do condomínio

\section{Estrutura do TCC}
Neste item você vai descrever como está constituída a monografia, indicando o
que será encontrado em cada uma das sessões seguintes.

\subsection{Classificação da Pesquisa}
Neste item será apresentada a classificação da pesquisa quanto aos objetivos
(exploratória, descritiva ou explicativa); aos procedimentos (Pesquisa
bibliográfica, Pesquisa documental, Pesquisa experimental, Estudo de caso
controle, Levantamento, Estudo de caso ou Estudo de campo) e ao método de
investigação científica (qualitativa ou quantitativa).
