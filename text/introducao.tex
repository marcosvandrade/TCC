\chapter{Introdução}
\label{chp:introduction}
%Faça aqui, uma introdução geral da área do conhecimento à qual o tema escolhido
%está ligado. 
O tema está diretamente ligado a área de desenvolvimento de software e será
desenvolvido uma Aplicação Web e Mobile utilizando a linguagem de programação
JavaScript. No frontend será utilizado o React, no backend será utilizado
o Nodejs e o banco de dados será o PostGresql. O React Native possibilitará a 
portabilidade para a plataforma mobile.
\section{Tema}
% A melhor forma de determinar o tema abordado é através de hipóteses. A hipótese
% consiste em uma afirmativa que você considera verdadeira e que vai provar ou
% buscar provar ao longo de seu trabalho. Outra forma é delimitando o problema em
% forma de uma pergunta de partida. Apresente uma visão geral do assunto que será
% abordado no trabalho.
O tema escolhido foi desenvolver um Sistema de Administração de Condomínios
personalizado. Atualmente, existem sistemas soluções personalizadas e acessíveis
para administração de condomínios?
\section{Problema}
% Dedique este tópico a esclarecer o que o pretende de fato com o seu esforço de
% pesquisa. Problema é a questão a ser respondida pelo trabalho, que motivou a sua
% realização. É uma questão que já tomou se formou em sua mente, derivada de
% teorias da área pesquisada e de sua observação sobre um fenômeno.  Normalmente
% se utilizam os subitens abaixo como meios de se determinar claramente os
% objetivos, o que também colabora para a delimitação do escopo do trabalho. Está
% estreitamente ligado ao objetivo geral, que, normalmente, consiste em encontrar
% a resposta para o problema de pesquisa.  O que você viu que é um problema que
% precisa de solução? É viável? Você consegue fazer? O problema é sempre uma
% dificuldade, uma lacuna.
Por morar em um condomínio, observei que o síndico aqui do meu bloco não possui
nenhum sistema para administração do condomínio e que mensalmente encaderna
mais ou menos umas 100(cem) folhas dentre comprovantes, prestações de conta,
notas fiscais, controle de empregados, etc.

Ao fazer uma breve pesquisa nos condomínios adjacentes percebi a mesma
problematica e percebi que para resolver essas dores poderia criar uma
oportunidade de negócio.

Ao conversar com o síndico do meu condomínio, tive o aceite de imediato e o 
meu condomínio será o piloto no desenvolvimento do meu negócio e em troca terá
custo zero o sistema que será criado, para o meu condomínio.

\subsection{Objetivo geral}
% É a resposta ao problema especificado acima, ou seja, aquilo que se pretende
% fazer e que, depois de atingido, estará concluído o trabalho.. Alguns verbos
% utilizados para determinar o objetivo geral: contribuir / facilitar / subsidiar
% / propor / clarear / permitir / agregar / compreender.
Desenvolver uma aplicação Web para administração de condomínios e que
inicialmente, para este TCC, terá como objetivo a criação do portal 
http://cliquesindico.com.br (que já encontra-se registrado e hospedado), onde
este portal principal será um "cartão de visitas" para o meu futuro negócio.

A idéia principal é modularizar a aplicação através de subdomínios, onde cada 
bloco ou condomínio terá um subdomínio único e com um sistema personalizado
de acordo com a necessidade de cada condomínio.

Para o meu condomínio, que será piloto neste projeto, foi criado o subdomínio
http://sqn306b.cliquesindico.com.br, onde na página principal terá uma página
de login, onde serão definidos alguns perfis, tais como administrador, gerente e
condomino, por exemplo. Após se logar na aplicação o usuário terá acesso a um
mural de avisos gerais do condomínio.

Enviar e-mail aos condominos com avisos, notificações, cobranças, etc

Criar uma página para prestação de contas mensal do condomínio.

Após a conclusão do sistema, o síndico e a diretoria do condomínio terão total
autonomia para inserir novos avisos, atualizar a prestação de contas e imprimir
relatórios.

\subsection{Objetivos específicos}
% Os objetivos específicos detalham os objetivos gerais através de etapas ou fases
% de pesquisa. Devem ser utilizados verbos no infinitivo, assinalando as ações
% propostas para alcançar o objetivo geral. Os verbos utilizados aqui são os de
% ação, que serão utilizados na metodologia.
Na página de login o usuário do sistema do condomínio terá a opção de se 
cadastrar, caso ainda não esteja cadastrado, e o login será realizado através
de e-mail e senha. O sistema notificará o síndico a fim de que valide o
cadastro ou revogue o acesso.

Será recolhido junto a administração do condomínio a documentação inicialmente
para que o sistema começe a ser desenvolvido, tais como, e-mail dos moradores,
avisos iniciais para inserção na página principal, modelo de ajuste de contas.

Implementar os diferentes perfis de acesso, de acordo com a real necessidade
do condomínio

\section{Estrutura do TCC}
Neste item você vai descrever como está constituída a monografia, indicando o
que será encontrado em cada uma das sessões seguintes.

\subsection{Classificação da Pesquisa}
Neste item será apresentada a classificação da pesquisa quanto aos objetivos
(exploratória, descritiva ou explicativa); aos procedimentos (Pesquisa
bibliográfica, Pesquisa documental, Pesquisa experimental, Estudo de caso
controle, Levantamento, Estudo de caso ou Estudo de campo) e ao método de
investigação científica (qualitativa ou quantitativa).
