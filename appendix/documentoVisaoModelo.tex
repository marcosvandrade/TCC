% This file was converted to LaTeX by Writer2LaTeX ver. 1.4
% see http://writer2latex.sourceforge.net for more info
\documentclass[a4paper]{article}
\usepackage[ascii]{inputenc}
\usepackage[T1]{fontenc}
\usepackage[english]{babel}
\usepackage{amsmath}
\usepackage{amssymb,amsfonts,textcomp}
\usepackage{color}
\usepackage{array}
\usepackage{supertabular}
\usepackage{hhline}
\usepackage{hyperref}
\hypersetup{pdftex, colorlinks=true, linkcolor=blue, citecolor=blue, filecolor=blue, urlcolor=blue, pdftitle=, pdfauthor=, pdfsubject=, pdfkeywords=}
\makeatletter
\newcommand\arraybslash{\let\\\@arraycr}
\makeatother
% List styles
\newcommand\liststyleWWNumiii{%
\renewcommand\labelitemi{${\bullet}$}
\renewcommand\labelitemii{${\circ}$}
\renewcommand\labelitemiii{${\blacksquare}$}
\renewcommand\labelitemiv{${\bullet}$}
}
\newcommand\liststyleWWNumii{%
\renewcommand\labelitemi{${\bullet}$}
\renewcommand\labelitemii{${\circ}$}
\renewcommand\labelitemiii{${\blacksquare}$}
\renewcommand\labelitemiv{${\bullet}$}
}
\newcommand\liststyleWWNumi{%
\renewcommand\labelitemi{${\bullet}$}
\renewcommand\labelitemii{${\circ}$}
\renewcommand\labelitemiii{${\blacksquare}$}
\renewcommand\labelitemiv{${\bullet}$}
}
% Page layout (geometry)
\setlength\voffset{-1in}
\setlength\hoffset{-1in}
\setlength\topmargin{1in}
\setlength\oddsidemargin{1in}
\setlength\textheight{9.6929in}
\setlength\textwidth{6.2681in}
\setlength\footskip{0.0cm}
\setlength\headheight{0cm}
\setlength\headsep{0cm}
% Footnote rule
\setlength{\skip\footins}{0.0469in}
\renewcommand\footnoterule{\vspace*{-0.0071in}\setlength\leftskip{0pt}\setlength\rightskip{0pt plus 1fil}\noindent\textcolor{black}{\rule{0.25\columnwidth}{0.0071in}}\vspace*{0.0398in}}
% Pages styles
\makeatletter
\newcommand\ps@Standard{
  \renewcommand\@oddhead{}
  \renewcommand\@evenhead{}
  \renewcommand\@oddfoot{}
  \renewcommand\@evenfoot{}
  \renewcommand\thepage{\arabic{page}}
}
\makeatother
\pagestyle{Standard}
\setlength\tabcolsep{1mm}
\renewcommand\arraystretch{1.3}
\title{}
\author{}
\date{}
\begin{document}
\clearpage\setcounter{page}{1}\pagestyle{Standard}

\bigskip

{\raggedleft
{\textless}Sistema de Administração de Condomíno Personalizado{\textgreater}
\par}

{\raggedleft
%%Vis\~ao
Visão
\par}


\bigskip

{\raggedleft
\textbf{Vers\~ao {\textless}1.0{\textgreater}}
\par}


\bigskip


\bigskip

\textit{\textcolor{blue}{[Nota: O gabarito a seguir \'e fornecido para utiliza\c{c}\~ao com o Rational Unified Process.
O texto em azul exibido entre colchetes e em it\'alico (style=InfoBlue) foi inclu\'ido para orientar o autor e deve ser
exclu\'ido antes da publica\c{c}\~ao do documento. Um par\'agrafo digitado ap\'os esse estilo ser\'a automaticamente
definido como normal (style=Body Text).]}}

\textit{\textcolor{blue}{[Para personalizar campos autom\'aticos no Microsoft Word (que exibem um segundo plano cinza
quando selecionados), selecione File{\textgreater}Properties e substitua os campos Title, Subject e Company pelas
informa\c{c}\~oes apropriadas para este documento. Depois de fechar o di\'alogo, os campos autom\'aticos podem ser
atualizados no documento inteiro, selecionando Edit{\textgreater}Select All (ou Ctrl-A) e pressionando F9 ou
simplesmente clique no campo e pressione F9. Esse procedimento dever\'a ser executado separadamente para os
Cabe\c{c}alhos e Rodap\'es. Alt-F9 alterna entre a exibi\c{c}\~ao de nomes de campos e do conte\'udo dos campos.
Consulte a Ajuda do Word para obter informa\c{c}\~oes adicionais sobre como trabalhar com campos.]}}

{\centering
\textbf{Hist\'orico da Revis\~ao}
\par}

\begin{flushleft}
\tablefirsthead{}
\tablehead{}
\tabletail{}
\tablelasttail{}
\begin{supertabular}{|m{1.4108598in}|m{0.70045984in}|m{2.3323598in}|m{1.5115598in}|}
\hline
~

{\centering \textbf{Data}\par}

~
 &
~

{\centering \textbf{Vers\~ao}\par}

~
 &
~

{\centering \textbf{Descri\c{c}\~ao \ \ \ \ \ \ } \ \ \ \ \ \ \par}

~
 &
~

{\centering \textbf{Autor}\par}

~
\\\hline
~

{\textless}dd/mmm/aa{\textgreater}

~
 &
~

{\textless}x.x{\textgreater}

~
 &
~

{\textless}detalhes{\textgreater}

~
 &
~

{\textless}nome{\textgreater}

~
\\\hline
~

~

~

~
 &
~

~

~

~
 &
~

~

~

~
 &
~

~

~

~
\\\hline
~

~

~

~
 &
~

~

~

~
 &
~

~

~

~
 &
~

~

~

~
\\\hline
~

~

~

~
 &
~

~

~

~
 &
~

~

~

~
 &
~

~

~

~
\\\hline
\end{supertabular}
\end{flushleft}

\bigskip

\textbf{SUM\'ARIO}

\setcounter{tocdepth}{10}
\renewcommand\contentsname{}
\tableofcontents

\bigskip


\bigskip


\bigskip

{\centering
Vis\~ao
\par}

\hypertarget{5w8r46f16jr3}{}\textbf{1.Introdu\c{c}\~ao}

\textit{\textcolor{blue}{[O objetivo deste documento \'e coletar, analisar e definir necessidades e recursos de alto
n\'ivel do }}\textcolor{blue}{{\textless}{\textless}Nome do
Sistema{\textgreater}{\textgreater}}\textit{\textcolor{blue}{. Ele est\'a focalizado nos recursos necess\'arios aos
investidores e usu\'arios de destino e }}\textbf{\textit{\textcolor{blue}{por que}}}\textit{\textcolor{blue}{ essas
necessidades existem. Os detalhes de como o }}\textcolor{blue}{{\textless}{\textless}Nome do
Sistema{\textgreater}{\textgreater}}\textit{\textcolor{blue}{ atende essas necessidades s\~ao explicados nas
especifica\c{c}\~oes de caso de uso e suplementares.]}}

\textit{\textcolor{blue}{[A introdu\c{c}\~ao do documento }}\textbf{\textit{\textcolor{blue}{Vis\~ao
}}}\textit{\textcolor{blue}{fornece uma vis\~ao geral de todo o documento. Ela inclui o objetivo, o escopo, as
defini\c{c}\~oes, os acr\^onimos, as abrevia\c{c}\~oes, as refer\^encias e a vis\~ao geral deste documento
}}\textbf{\textit{\textcolor{blue}{Vis\~ao}}}\textit{\textcolor{blue}{.]}}

\hypertarget{srl4xio0egtt}{}\textbf{1.1Objetivo}

\textit{\textcolor{blue}{[Especifique o objetivo deste documento
}}\textbf{\textit{\textcolor{blue}{Vis\~ao.}}}\textit{\textcolor{blue}{]}}

\hypertarget{tvu2oa77sfsk}{}\textbf{1.2Escopo}

\textit{\textcolor{blue}{[Uma breve descri\c{c}\~ao do escopo deste documento
}}\textbf{\textit{\textcolor{blue}{Vis\~ao}}}\textit{\textcolor{blue}{; a qual(is) Projeto(s) ele est\'a associado e
tudo mais que seja afetado ou influenciado por este documento.]}}

\hypertarget{32ir7uiczi8s}{}\textbf{1.3Defini\c{c}\~oes, Acr\^onimos e Abrevia\c{c}\~oes}

\textit{\textcolor{blue}{[Esta subse\c{c}\~ao fornece as defini\c{c}\~oes de todos os termos, acr\^onimos e
abrevia\c{c}\~oes requeridos para interpretar adequadamente o documento
}}\textbf{\textit{\textcolor{blue}{Vis\~ao}}}\textit{\textcolor{blue}{. Essas informa\c{c}\~oes podem ser fornecidas em
rela\c{c}\~ao ao Gloss\'ario do projeto.]}}

\hypertarget{jdov52ki4krl}{}\textbf{1.4Refer\^encias}

\textit{\textcolor{blue}{[Esta subse\c{c}\~ao fornece uma lista completa de todos os documentos mencionados em outra
parte no documento }}\textbf{\textit{\textcolor{blue}{Vis\~ao}}}\textit{\textcolor{blue}{. Identifique cada documento
por t\'itulo, n\'umero do relat\'orio se aplic\'avel, data e organiza\c{c}\~ao da publica\c{c}\~ao. Especifique as
origens a partir das quais as refer\^encias podem ser obtidas. Essas informa\c{c}\~oes podem ser fornecidas por um
anexo ou outro documento.]}}

\hypertarget{vhgkzpq1352v}{}\textbf{1.5Vis\~ao Geral}

\textit{\textcolor{blue}{[Esta subse\c{c}\~ao descreve o que o restante do documento
}}\textbf{\textit{\textcolor{blue}{Vis\~ao }}}\textit{\textcolor{blue}{cont\'em e explica como o documento \'e
organizado.]}}

\hypertarget{h594f1srb451}{}\textbf{2.Posicionamento}

\hypertarget{ullot36our2x}{}\textbf{2.1Oportunidade de Neg\'ocio}

\textit{\textcolor{blue}{[Descreva resumidamente a oportunidade de neg\'ocio que est\'a sendo atendida por este
projeto.]}}

\hypertarget{ms4flj8sfg29}{}\textbf{2.2Declara\c{c}\~ao do Problema}

\textit{\textcolor{blue}{[Forne\c{c}a uma declara\c{c}\~ao resumindo o problema que est\'a sendo resolvido por este
projeto. O formato a seguir pode ser utilizado:]}}


\bigskip


\bigskip

\begin{flushleft}
\tablefirsthead{}
\tablehead{}
\tabletail{}
\tablelasttail{}
\begin{supertabular}{m{2.1018598in}|m{3.7261598in}|}
\hhline{~-}
~

\ \ \ \ \ \ \ \ O problema de

~
 &
~

\ \ \ \ \ \ \ \ \textit{\textcolor{blue}{[descreva o problema]}}

~
\\\hhline{~-}
~

\ \ \ \ \ \ \ \ afeta

~
 &
~

\ \ \ \ \ \ \ \ \textit{\textcolor{blue}{[os investidores afetados pelo problema]}}

~
\\\hhline{~-}
~

\ \ \ \ \ \ \ \ o impacto \'e o seguinte

~
 &
~

\ \ \ \ \ \ \ \ \textit{\textcolor{blue}{[qual \'e o impacto do problema?]}}

~
\\\hhline{~-}
~

uma \ \ \ \ \ \ \ \ solu\c{c}\~ao bem-sucedida seria

~
 &
~

\ \ \ \ \ \ \ \ \textit{\textcolor{blue}{[liste alguns benef\'icios chave de uma \ \ \ \ \ \ \ \ solu\c{c}\~ao
bem-sucedida]}}

~
\\\hhline{~-}
\end{supertabular}
\end{flushleft}
\hypertarget{akoq5khmv06x}{}\textbf{2.3Declara\c{c}\~ao da Posi\c{c}\~ao do Produto}

\textit{\textcolor{blue}{[Forne\c{c}a uma declara\c{c}\~ao geral resumindo, no n\'ivel mais alto, a posi\c{c}\~ao
exclusiva que o produto pretende ocupar no marketplace. O formato a seguir pode ser utilizado:]}}


\bigskip


\bigskip

\begin{flushleft}
\tablefirsthead{}
\tablehead{}
\tabletail{}
\tablelasttail{}
\begin{supertabular}{m{1.9768598in}|m{3.8511598in}|}
\hhline{~-}
~

\ \ \ \ \ \ \ \ Para

~
 &
~

\ \ \ \ \ \ \ \ \textit{\textcolor{blue}{[cliente de destino]}}

~
\\\hhline{~-}
~

\ \ \ \ \ \ \ \ Quem

~
 &
~

\ \ \ \ \ \ \ \ \textit{\textcolor{blue}{[declara\c{c}\~ao da necessidade ou \ \ \ \ \ \ \ \ oportunidade]}}

~
\\\hhline{~-}
~

\ \ \ \ \ \ \ \ O (nome do produto)

~
 &
~

\ \ \ \ \ \ \ \ \textcolor{blue}{ }\textit{\textcolor{blue}{\'e um [categoria do produto]}}

~
\\\hhline{~-}
~

\ \ \ \ \ \ \ \ Que

~
 &
~

\ \ \ \ \ \ \ \ \textit{\textcolor{blue}{[declara\c{c}\~ao do benef\'icio chave, isto \ \ \ \ \ \ \ \ \'e, o motivo que
leva a comprar]}}

~
\\\hhline{~-}
~

\ \ \ \ \ \ \ \ Diferente

~
 &
~

\ \ \ \ \ \ \ \ \textit{\textcolor{blue}{[principal alternativa competitiva]}}

~
\\\hhline{~-}
~

Nosso \ \ \ \ \ \ \ \ produto

~
 &
~

\ \ \ \ \ \ \ \ \textit{\textcolor{blue}{[declara\c{c}\~ao da diferencia\c{c}\~ao \ \ \ \ \ \ \ \ principal]}}

~
\\\hhline{~-}
\end{supertabular}
\end{flushleft}
\textit{\textcolor{blue}{[Uma declara\c{c}\~ao da posi\c{c}\~ao do produto comunica a inten\c{c}\~ao do aplicativo e a
import\^ancia do projeto para todo o pessoal interessado.]}}

\hypertarget{3senbrel00q1}{}\textbf{3.Descri\c{c}\~oes do Investidor e do Usu\'ario}

\textit{\textcolor{blue}{[Para fornecer produtos e servi\c{c}os que efetivamente atendam as necessidades reais de seus
investidores e usu\'arios, \'e necess\'ario identificar e envolver todos os investidores como parte do processo de
Modelagem de Requisitos. Voc\^e deve tamb\'em identificar os usu\'arios do sistema e garantir que a comunidade de
investidores os represente adequadamente. Esta se\c{c}\~ao fornece um perfil dos investidores e usu\'arios envolvidos
no projeto e os problemas chave que eles observam para que sejam tratados pela solu\c{c}\~ao proposta. N\~ao descreve
os pedidos ou requisitos espec\'ificos uma vez que estes s\~ao capturados em um artefato separado de pedidos do
investidor. Em vez disso, fornece o segundo plano e a justificativa de por que os requisitos s\~ao necess\'arios.]}}

\hypertarget{p0ns12v6bh5j}{}\textbf{3.1Demogr\'aficos de Mercado}

\textit{\textcolor{blue}{[Resuma os demogr\'aficos chave de mercado que motivam as decis\~oes do produto. Descreva e
posicione os segmentos de mercado de destino. Fa\c{c}a uma estimativa do tamanho e do crescimento do mercado utilizando
o n\'umero de usu\'arios potenciais ou o valor em dinheiro que seus clientes gastam tentando atender as necessidades
que seu produto ou aprimoramento poderia suprir. Reveja as principais tend\^encias e tecnologias do segmento de
mercado. Responda estas perguntas estrat\'egicas:}}

\textcolor{blue}{{\textbullet} }\textit{\textcolor{blue}{\ \ Qual \'e a reputa\c{c}\~ao da sua organiza\c{c}\~ao nesses
mercados?}}

\textcolor{blue}{{\textbullet}}\textit{\textcolor{blue}{\ \ Qual voc\^e gostaria que fosse?}}

\textcolor{blue}{{\textbullet} }\textit{\textcolor{blue}{\ \ Como este produto ou servi\c{c}o suporta suas metas?]}}

\hypertarget{n8gm6kj3eezs}{}\textbf{3.2Resumo do Investidor}

\textit{\textcolor{blue}{[H\'a v\'arios investidores com interesse no desenvolvimento e nem todos eles s\~ao usu\'arios
finais. Apresente uma lista de resumo desses investidores n\~ao-usu\'arios. (Os usu\'arios est\~ao resumidos na
se\c{c}\~ao 3.3.)]}}


\bigskip


\bigskip

\begin{flushleft}
\tablefirsthead{}
\tablehead{}
\tabletail{}
\tablelasttail{}
\begin{supertabular}{m{1.3545599in}m{1.8629599in}m{2.81706in}}
~

\textbf{Nome}

~
 &
~

\textbf{Descri\c{c}\~ao}

~
 &
~

\textbf{Responsabilidades}

~
\\\hline
\multicolumn{1}{|m{1.3545599in}|}{~

\ \ \ \ \ \ \ \ \textit{\textcolor{blue}{[Nomeie o tipo do investidor.]}}

~
} &
\multicolumn{1}{m{1.8629599in}|}{~

\ \ \ \ \ \ \ \ \textit{\textcolor{blue}{[Descreva resumidamente o investidor.]}}

~
} &
\multicolumn{1}{m{2.81706in}|}{~

\ \ \ \ \ \ \ \ \textit{\textcolor{blue}{[Resuma as principais responsabilidades \ \ \ \ \ \ \ \ do investidor com
rela\c{c}\~ao ao sistema que est\'a sendo \ \ \ \ \ \ \ \ desenvolvido, isto \'e, seu interesse como um investidor. Por
\ \ \ \ \ \ \ \ exemplo, este investidor:}}

~

\liststyleWWNumiii
\begin{itemize}
\item \ \ \ \ \ \ \ \ \ \ \newline
 \textit{\textcolor{blue}{garante \ \ \ \ \ \ \ \ \ \ que ser\'a poss\'ivel manter o sistema\newline
}} \ \ \ \ \ \ \ \ \ \ 
\item \textit{\textcolor{blue}{garante \ \ \ \ \ \ \ \ \ \ que haver\'a uma demanda de mercado para os recursos do
produto\newline
}} \ \ \ \ \ \ \ \ \ \ 
\item \textit{\textcolor{blue}{monitora \ \ \ \ \ \ \ \ \ \ o andamento do projeto\newline
}} \ \ \ \ \ \ \ \ \ \ 
\item \textit{\textcolor{blue}{aprova \ \ \ \ \ \ \ \ \ \ o fundo\newline
}} \ \ \ \ \ \ \ \ \ \ 
\item \textit{\textcolor{blue}{e \ \ \ \ \ \ \ \ \ \ assim por diante]\newline
}} \ \ \ \ \ \ \ \ 
\end{itemize}
~
}\\\hline
\end{supertabular}
\end{flushleft}
\hypertarget{b49lq6ax8znp}{}\textbf{3.3Resumo de Usu\'arios}

\textit{\textcolor{blue}{[Apresente uma lista de resumo de todos os usu\'arios identificados.]}}


\bigskip


\bigskip

\begin{flushleft}
\tablefirsthead{}
\tablehead{}
\tabletail{}
\tablelasttail{}
\begin{supertabular}{m{0.72125983in}m{1.2726599in}m{2.16776in}m{1.7934599in}}
~

\textbf{Nome}

~
 &
~

\textbf{Descri\c{c}\~ao}

~
 &
~

\textbf{Responsabilidades}

~
 &
~

\textbf{Investidor}

~
\\\hline
\multicolumn{1}{|m{0.72125983in}|}{~

\ \ \ \ \ \ \ \ \textit{\textcolor{blue}{[Nomeie o tipo de usu\'ario.]}}

~
} &
\multicolumn{1}{m{1.2726599in}|}{~

\ \ \ \ \ \ \ \ \textit{\textcolor{blue}{[Descreva resumidamente o que \ \ \ \ \ \ \ \ representam com rela\c{c}\~ao ao
sistema.]}}

~
} &
\multicolumn{1}{m{2.16776in}|}{~

\ \ \ \ \ \ \ \ \textit{\textcolor{blue}{[Liste as principais responsabilidades \ \ \ \ \ \ \ \ do usu\'ario com
rela\c{c}\~ao ao sistema que est\'a sendo \ \ \ \ \ \ \ \ desenvolvido, por exemplo:}}

~

\liststyleWWNumii
\begin{itemize}
\item \ \ \ \ \ \ \ \ \ \ \newline
 \textit{\textcolor{blue}{captura \ \ \ \ \ \ \ \ \ \ detalhes\newline
}} \ \ \ \ \ \ \ \ \ \ 
\item \textit{\textcolor{blue}{produz \ \ \ \ \ \ \ \ \ \ relat\'orios\newline
}} \ \ \ \ \ \ \ \ \ \ 
\item \textit{\textcolor{blue}{coordena \ \ \ \ \ \ \ \ \ \ o trabalho\newline
}} \ \ \ \ \ \ \ \ \ \ 
\item \textit{\textcolor{blue}{e \ \ \ \ \ \ \ \ \ \ assim por diante]\newline
}} \ \ \ \ \ \ \ \ 
\end{itemize}
~
} &
\multicolumn{1}{m{1.7934599in}|}{~

\ \ \ \ \ \ \ \ \textit{\textcolor{blue}{[Se o usu\'ario n\~ao for diretamente \ \ \ \ \ \ \ \ representado, identifique
qual investidor \'e respons\'avel por \ \ \ \ \ \ \ \ representar o interesse do usu\'ario.]}}

~
}\\\hline
\end{supertabular}
\end{flushleft}

\bigskip


\bigskip

\hypertarget{wbxdw99bjw1b}{}\textbf{3.4Ambiente do Usu\'ario}

\textit{\textcolor{blue}{[Detalhe o ambiente de trabalho do usu\'ario de destino. A seguir, s\~ao apresentadas algumas
sugest\~oes:}}

\liststyleWWNumi
\begin{itemize}
\item \ \ \newline
 \textit{\textcolor{blue}{N\'umero \ \ de pessoas envolvidas na conclus\~ao da tarefa? Isso est\'a mudando?\newline
}} \ \ 
\item \textit{\textcolor{blue}{Qual \ \ \'e a dura\c{c}\~ao de um ciclo de tarefa? Per\'iodo de tempo gasto em \ \ cada
atividade? Isso est\'a mudando?\newline
}} \ \ 
\item \textit{\textcolor{blue}{Alguma \ \ restri\c{c}\~ao ambiental exclusiva: m\'ovel, ao ar livre, em v\^oo e
\ \ assim por diante?\newline
}} \ \ 
\item \textit{\textcolor{blue}{Quais \ \ plataformas de sistemas est\~ao em uso hoje? Plataformas futuras?\newline
}} \ \ 
\item \textit{\textcolor{blue}{Que \ \ outros aplicativos est\~ao em uso? Seu aplicativo precisa se integrar \ \ a
eles?\newline
}}
\end{itemize}
\textit{\textcolor{blue}{Aqui \'e onde as extra\c{c}\~oes do Modelo de Neg\'ocio podem ser inclu\'idas para esbo\c{c}ar
a tarefa e as fun\c{c}\~oes envolvidas e assim por diante.]}}

\hypertarget{9h7ts32xj51n}{}\textbf{3.5Perfis do Investidor}

\textit{\textcolor{blue}{[Descreva cada investidor no sistema aqui preenchendo a seguinte tabela para cada investidor.
Lembre-se de que os tipos de investidor podem ser t\~ao diferentes quanto usu\'arios, departamentos e desenvolvedores
t\'ecnicos. Um perfil completo cobriria os seguintes t\'opicos para cada tipo de investidor.]}}

{\color[rgb]{0.2627451,0.2627451,0.2627451}
\hypertarget{qn8n5a4zkm95}{}\textbf{\textcolor{black}{3.5.1{\textless}Nome do Investidor{\textgreater}}}}


\bigskip


\bigskip

\begin{flushleft}
\tablefirsthead{}
\tablehead{}
\tabletail{}
\tablelasttail{}
\begin{supertabular}{|m{1.2684599in}|m{4.84416in}|}
\hline
~

\ \ \ \ \ \ \ \ \textbf{Representante}

~
 &
~

\ \ \ \ \ \ \ \ \textit{\textcolor{blue}{[Quem \'e o representante do investidor \ \ \ \ \ \ \ \ para o projeto?
(Opcional se documentado em outro lugar.) O que \ \ \ \ \ \ \ \ queremos aqui s\~ao nomes.]}}

~
\\\hline
~

\ \ \ \ \ \ \ \ \textbf{Descri\c{c}\~ao}

~
 &
~

\ \ \ \ \ \ \ \ \textit{\textcolor{blue}{[Uma breve descri\c{c}\~ao do tipo de \ \ \ \ \ \ \ \ investidor.]}}

~
\\\hline
~

\ \ \ \ \ \ \ \ \textbf{Tipo}

~
 &
~

\ \ \ \ \ \ \ \ \textit{\textcolor{blue}{[Qualifique o conhecimento do \ \ \ \ \ \ \ \ investidor, o background
t\'ecnico e o grau de sofistica\c{c}\~ao---isto \ \ \ \ \ \ \ \ \'e, guru, neg\'ocios, especialista, usu\'ario casual e
assim por \ \ \ \ \ \ \ \ diante.]}}

~
\\\hline
~

\ \ \ \ \ \ \ \ \textbf{Responsabilidades}

~
 &
~

\ \ \ \ \ \ \ \ \textit{\textcolor{blue}{[Liste as principais responsabilidades \ \ \ \ \ \ \ \ do investidor com
rela\c{c}\~ao ao sistema que est\'a sendo \ \ \ \ \ \ \ \ desenvolvido---isto \'e, seu interesse como investidor.]}}

~
\\\hline
~

\ \ \ \ \ \ \ \ \textbf{Crit\'erios de \^Exito}

~
 &
~

\ \ \ \ \ \ \ \ \textit{\textcolor{blue}{[Como o investidor define o \^exito? }}\ \ \ \ \ \ \ \ 

~

\ \ \ \ \ \ \ \ \textit{\textcolor{blue}{Como o investidor \'e recompensado?]}}

~
\\\hline
~

\ \ \ \ \ \ \ \ \textbf{Envolvimento}

~
 &
~

\ \ \ \ \ \ \ \ \textit{\textcolor{blue}{[Como o investidor est\'a envolvido no \ \ \ \ \ \ \ \ projeto? Relacione, onde
poss\'ivel, com fun\c{c}\~oes do Rational \ \ \ \ \ \ \ \ Unified Process---isto \'e, Revisor de Requisitos e assim por
\ \ \ \ \ \ \ \ diante.]}}

~
\\\hline
~

\ \ \ \ \ \ \ \ \textbf{Distribu\'iveis}

~
 &
~

\ \ \ \ \ \ \ \ \textit{\textcolor{blue}{[H\'a algum distribu\'ivel adicional \ \ \ \ \ \ \ \ requerido pelo investidor?
Podem ser distribu\'iveis ou sa\'idas do \ \ \ \ \ \ \ \ projeto do sistema em desenvolvimento.]}}

~
\\\hline
~

\ \ \ \ \ \ \ \ \textbf{Coment\'arios / Problemas}

~
 &
~

\ \ \ \ \ \ \ \ \textit{\textcolor{blue}{[Problemas que interferem no \^exito e \ \ \ \ \ \ \ \ qualquer outra
informa\c{c}\~ao relevante devem ser colocados aqui.]}}

~
\\\hline
\end{supertabular}
\end{flushleft}

\bigskip


\bigskip

\hypertarget{kgshcf56lccx}{}\textbf{3.6Perfis de Usu\'arios}

\textit{\textcolor{blue}{[Descreva cada usu\'ario exclusivo do sistema aqui preenchendo a seguinte tabela para cada tipo
de usu\'ario. Lembre-se de que os tipos de usu\'arios podem ser t\~ao diferentes quanto gurus e aprendizes. Por
exemplo, um guru pode precisar de uma ferramenta sofisticada, flex\'ivel com suporte de plataforma cruzada enquanto que
um aprendiz pode precisar de uma ferramenta f\'acil de utilizar e simples. Um perfil completo deve cobrir os seguintes
t\'opicos para cada tipo de usu\'ario.]}}

{\color[rgb]{0.2627451,0.2627451,0.2627451}
\hypertarget{8f1gl1now1kh}{}\textbf{\textcolor{black}{3.6.1{\textless}Nome do Usu\'ario{\textgreater}}}}


\bigskip


\bigskip


\bigskip


\bigskip

\begin{flushleft}
\tablefirsthead{}
\tablehead{}
\tabletail{}
\tablelasttail{}
\begin{supertabular}{|m{1.2684599in}|m{4.84416in}|}
\hline
~

\ \ \ \ \ \ \ \ \textbf{Representante}

~
 &
~

\ \ \ \ \ \ \ \ \textit{\textcolor{blue}{[Quem \'e o representante do usu\'ario \ \ \ \ \ \ \ \ para o projeto?
(Opcional se documentado em \ \ \ \ \ \ \ \ outro lugar.) Isso, freq\"uentemente, refere-se ao
\ \ \ \ \ \ \ \ Investidor que representa o conjunto de usu\'arios, por exemplo, \ \ \ \ \ \ \ \ Investidor:
Investidor1.]}}

~
\\\hline
~

\ \ \ \ \ \ \ \ \textbf{Descri\c{c}\~ao}

~
 &
~

\ \ \ \ \ \ \ \ \textit{\textcolor{blue}{[Uma breve descri\c{c}\~ao do tipo de \ \ \ \ \ \ \ \ usu\'ario.]}}

~
\\\hline
~

\ \ \ \ \ \ \ \ \textbf{Tipo}

~
 &
~

\ \ \ \ \ \ \ \ \textit{\textcolor{blue}{[Qualifique o conhecimento do usu\'ario, \ \ \ \ \ \ \ \ o background t\'ecnico
e o grau de sofistica\c{c}\~ao---isto \'e, guru, \ \ \ \ \ \ \ \ usu\'ario casual e assim por diante.]
}}\ \ \ \ \ \ \ \ 

~
\\\hline
~

\ \ \ \ \ \ \ \ \textbf{Responsabilidades}

~
 &
~

\ \ \ \ \ \ \ \ \textit{\textcolor{blue}{[Liste as principais responsabilidades \ \ \ \ \ \ \ \ do usu\'ario com
rela\c{c}\~ao ao sistema que est\'a sendo \ \ \ \ \ \ \ \ desenvolvido--- isto \'e, captura detalhes, produz
relat\'orios, \ \ \ \ \ \ \ \ coordena o trabalho e assim por diante.]}}

~
\\\hline
~

\ \ \ \ \ \ \ \ \textbf{Crit\'erios de \^Exito}

~
 &
~

\ \ \ \ \ \ \ \ \textit{\textcolor{blue}{[Como o usu\'ario define o \^exito?}}

~

\ \ \ \ \ \ \ \ \textcolor{blue}{ }\textit{\textcolor{blue}{Como o usu\'ario \'e recompensado?]}}

~
\\\hline
~

\ \ \ \ \ \ \ \ \textbf{Envolvimento}

~
 &
~

\ \ \ \ \ \ \ \ \textit{\textcolor{blue}{[Como o usu\'ario est\'a envolvido no \ \ \ \ \ \ \ \ projeto? Relacione, onde
poss\'ivel, com fun\c{c}\~oes do Rational \ \ \ \ \ \ \ \ Unified Process---isto \'e, Revisor de Requisitos e assim por
\ \ \ \ \ \ \ \ diante.]}}

~
\\\hline
~

\ \ \ \ \ \ \ \ \textbf{Distribu\'iveis}

~
 &
~

\ \ \ \ \ \ \ \ \textit{\textcolor{blue}{[H\'a algum distribu\'ivel que o usu\'ario \ \ \ \ \ \ \ \ produz e, se houver,
para quem?]}}

~
\\\hline
~

\ \ \ \ \ \ \ \ \textbf{Coment\'arios / Problemas}

~
 &
~

\ \ \ \ \ \ \ \ \textit{\textcolor{blue}{[Problemas que interferem no \^exito e \ \ \ \ \ \ \ \ qualquer outra
informa\c{c}\~ao relevante devem ser colocados aqui. \ \ \ \ \ \ \ \ Esses incluiriam tend\^encias que tornam o
trabalho do usu\'ario \ \ \ \ \ \ \ \ mais f\'acil ou mais dif\'icil.]}}

~
\\\hline
\end{supertabular}
\end{flushleft}

\bigskip


\bigskip

\hypertarget{csn2idtskkzk}{}\textbf{3.7Necessidades Principais do Investidor ou Usu\'ario}

\textit{\textcolor{blue}{[Liste os problemas chave com solu\c{c}\~oes existentes conforme observado pelo investidor ou
usu\'ario. Explique as seguintes quest\~oes para cada problema:}}

\textcolor{blue}{{\textbullet}}\textit{\textcolor{blue}{\ \ Quais s\~ao os motivos para este problema?}}

\textcolor{blue}{{\textbullet}}\textit{\textcolor{blue}{\ \ Como ele \'e resolvido agora?}}

\textcolor{blue}{{\textbullet}}\textit{\textcolor{blue}{\ \ Quais solu\c{c}\~oes o investidor ou o usu\'ario deseja?]}}

\textit{\textcolor{blue}{[\'E importante entender a import\^ancia
}}\textbf{\textit{\textcolor{blue}{relativa}}}\textit{\textcolor{blue}{ que o investidor ou o usu\'ario coloca em
resolver cada problema. T\'ecnicas de classifica\c{c}\~ao e vota\c{c}\~ao acumulativa indicam problemas que
}}\textbf{\textit{\textcolor{blue}{devem}}}\textit{\textcolor{blue}{ ser resolvidos contra problemas que eles gostariam
que fossem tratados.}}

\textit{\textcolor{blue}{Preencha a tabela a seguir---se estiver utilizando o Rational RequisitePro para capturar as
Necessidades, isso poderia ser uma extra\c{c}\~ao ou relat\'orio dessa ferramenta.]}}

\begin{flushleft}
\tablefirsthead{}
\tablehead{}
\tabletail{}
\tablelasttail{}
\begin{supertabular}{m{0.78095984in}m{0.6809598in}m{0.60315984in}m{0.5698598in}m{2.9073598in}m{0.25595984in}}
~

\textbf{Necessidade}

~
 &
~

\textbf{Prioridade}

~
 &
~

\textbf{Assuntos}

~
 &
~

\textbf{Solu\c{c}\~ao \ \ \ \ \ \ Atual}

~
 &
\multicolumn{2}{m{3.2420597in}}{~

\textbf{Solu\c{c}\~oes \ \ \ \ \ \ Propostas}

~
}\\\hline
\multicolumn{1}{|m{0.78095984in}|}{~

Mensagens \ \ \ \ \ \ de difus\~ao

~
} &
\multicolumn{1}{m{0.6809598in}|}{~

~

~

~
} &
\multicolumn{1}{m{0.60315984in}|}{~

~

~

~
} &
\multicolumn{2}{m{3.5559597in}|}{~

~

~

~
} &
\multicolumn{1}{m{0.25595984in}|}{~

~

~

~
}\\\hline
\end{supertabular}
\end{flushleft}

\bigskip


\bigskip

\hypertarget{ojfvyd3ys12u}{}\textbf{3.8Alternativas e Competi\c{c}\~ao}

\textit{\textcolor{blue}{[Identifique alternativas que o investidor observa como dispon\'iveis. Estas podem incluir
comprar o produto de um concorrente, construir uma solu\c{c}\~ao pr\'opria ou simplesmente manter o status quo. Liste
todas as op\c{c}\~oes competitivas conhecidas que existem ou que podem se tornar dispon\'iveis. Inclua os principais
pontos fortes e fracos de cada concorrente conforme observado pelo investidor ou usu\'ario final.]}}

{\color[rgb]{0.2627451,0.2627451,0.2627451}
\hypertarget{528oqfd97zah}{}\textbf{\textcolor{black}{3.8.1{\textless}aCompetitor{\textgreater}}}}

{\color[rgb]{0.2627451,0.2627451,0.2627451}
\hypertarget{t54ljqskjpsf}{}\textbf{\textcolor{black}{3.8.2{\textless}anotherCompetitor{\textgreater}}}}

\hypertarget{lggugq9mkeu5}{}\textbf{4.Vis\~ao Geral do Produto}

\textit{\textcolor{blue}{[Esta se\c{c}\~ao fornece uma visualiza\c{c}\~ao de alto n\'ivel dos recursos do produto,
interfaces com outros aplicativos e configura\c{c}\~oes do sistema. Esta se\c{c}\~ao, geralmente, consiste em tr\^es
subse\c{c}\~oes, como segue:}}

\textcolor{blue}{{\textbullet}}\textit{\textcolor{blue}{\ \ Perspectiva do produto}}

\textcolor{blue}{{\textbullet}}\textit{\textcolor{blue}{\ \ Fun\c{c}\~oes do produto}}

\textcolor{blue}{{\textbullet}}\textit{\textcolor{blue}{\ \ Premissas e depend\^encias]}}

\hypertarget{x6rte7p40tn8}{}\textbf{4.1Perspectiva do Produto}

\textit{\textcolor{blue}{[Esta subse\c{c}\~ao do documento
}}\textbf{\textit{\textcolor{blue}{Vis\~ao}}}\textit{\textcolor{blue}{ coloca o produto em perspectiva com outros
produtos relacionados e o ambiente do usu\'ario. Se o produto for totalmente independente, declare isso aqui. Se o
produto for um componente de um sistema maior, esta subse\c{c}\~ao dever\'a relatar como esses sistemas interagem e
dever\'a identificar as interfaces relevantes entre os sistemas. Uma maneira f\'acil de exibir os principais
componentes do sistema maior, interconex\~oes e interfaces externas \'e com um diagrama de bloco.]}}

\hypertarget{p8wle8ackrb6}{}\textbf{4.2Resumo de Recursos}

\textit{\textcolor{blue}{[Resuma os principais benef\'icios e recursos que o produto fornecer\'a. Por exemplo, um
documento }}\textbf{\textit{\textcolor{blue}{Vis\~ao}}}\textit{\textcolor{blue}{ para um sistema de suporte ao cliente
pode utilizar esta parte para tratar da documenta\c{c}\~ao, da rota e do relat\'orio de status do problema sem
mencionar a quantidade de detalhes que cada uma dessas fun\c{c}\~oes requer.}}

\textit{\textcolor{blue}{Organize as fun\c{c}\~oes de modo que a lista seja compreens\'ivel para o cliente e para
qualquer outra pessoa que esteja lendo o documento pela primeira vez. Uma tabela simples listando os principais
benef\'icios e seus recursos de suporte pode ser suficiente. Por exemplo:]}}

Tabela 4-1 Sistema de Suporte ao Cliente


\bigskip


\bigskip


\bigskip

\begin{flushleft}
\tablefirsthead{}
\tablehead{}
\tabletail{}
\tablelasttail{}
\begin{supertabular}{|m{2.25806in}|m{2.74696in}|}
\hline
~

\ \ \ \ \ \ \ \ \ \ \textbf{Benef\'icio do Cliente}

~
 &
~

\ \ \ \ \ \ \ \ \ \ \textbf{Recursos de Suporte}

~
\\\hline
~

\ \ \ \ \ \ \ \ \ \ A nova equipe de suporte pode rapidamente alcan\c{c}ar velocidade.

~
 &
~

\ \ \ \ \ \ \ \ \ \ A base de conhecimento ajuda o pessoal de suporte a identificar \ \ \ \ \ \ \ \ \ \ rapidamente
corre\c{c}\~oes conhecidas e solu\c{c}\~oes alternativas.

~
\\\hline
~

\ \ \ \ \ \ \ \ \ \ A satisfa\c{c}\~ao do cliente \'e aprimorada porque nada \'e deixado \ \ \ \ \ \ \ \ \ \ para
tr\'as.

~
 &
~

\ \ \ \ \ \ \ \ \ \ Os problemas s\~ao exclusivamente relacionados por itens, \ \ \ \ \ \ \ \ \ \ classificados e
rastreados em todo o processo de resolu\c{c}\~ao. A \ \ \ \ \ \ \ \ \ \ notifica\c{c}\~ao autom\'atica ocorre para
problemas de qualquer \ \ \ \ \ \ \ \ \ \ per\'iodo.

~
\\\hline
~

\ \ \ \ \ \ \ \ \ \ O gerenciamento pode identificar \'areas de problemas e \ \ \ \ \ \ \ \ \ \ determinar a carga de
trabalho da equipe.

~
 &
~

\ \ \ \ \ \ \ \ \ \ Os relat\'orios de tend\^encia e distribui\c{c}\~ao permitem uma \ \ \ \ \ \ \ \ \ \ revis\~ao de
alto n\'ivel do status do problema.

~
\\\hline
~

\ \ \ \ \ \ \ \ \ \ As equipes de suporte distribu\'idas podem trabalhar juntas para \ \ \ \ \ \ \ \ \ \ resolver
problemas.

~
 &
~

\ \ \ \ \ \ \ \ \ \ O servidor de replica\c{c}\~ao permite que informa\c{c}\~oes do banco de \ \ \ \ \ \ \ \ \ \ dados
atual sejam compartilhadas na empresa.

~
\\\hline
~

\ \ \ \ \ \ \ \ \ \ Os clientes podem se ajudar, baixando os custos do suporte e \ \ \ \ \ \ \ \ \ \ aprimorando o tempo
de resposta.

~
 &
~

\ \ \ \ \ \ \ \ \ \ A base de conhecimento pode ser disponibilizada atrav\'es da \ \ \ \ \ \ \ \ \ \ Internet. Inclui
recursos de pesquisa de hipertexto e mecanismo \ \ \ \ \ \ \ \ \ \ de consulta gr\'afica.

~
\\\hline
\end{supertabular}
\end{flushleft}

\bigskip

\hypertarget{ryhuvdw0p8rj}{}\textbf{4.3Premissas e Depend\^encias}

\textit{\textcolor{blue}{[Liste cada um dos fatores que afetam os recursos declarados no documento
}}\textbf{\textit{\textcolor{blue}{Vis\~ao}}}\textit{\textcolor{blue}{. Liste premissas que, se alteradas, mudar\~ao o
documento }}\textbf{\textit{\textcolor{blue}{Vis\~ao}}}\textit{\textcolor{blue}{. Por exemplo, uma premissa pode
declarar que um sistema operacional espec\'ifico estar\'a dispon\'ivel para o hardware designado para o produto de
software. Se o sistema operacional n\~ao estiver dispon\'ivel, o documento
}}\textbf{\textit{\textcolor{blue}{Vis\~ao}}}\textit{\textcolor{blue}{ precisar\'a ser alterado.]}}

\hypertarget{iqoys7sdrgpk}{}\textbf{4.4Custo e Pre\c{c}o}

\textit{\textcolor{blue}{[Para produtos vendidos para clientes externos e para muitos aplicativos internos, os problemas
de custo e pre\c{c}o podem impactar diretamente a defini\c{c}\~ao e a implementa\c{c}\~ao do aplicativo. Nesta
se\c{c}\~ao, registre quaisquer restri\c{c}\~oes de custo e pre\c{c}o que sejam relevantes. Por exemplo, custos de
distribui\c{c}\~ao, (\# de disquetes, \# de CD-ROMs, controle do CD) ou outras restri\c{c}\~oes de custo de mercadorias
vendidas (manuais, pacote) podem ser materiais para o \^exito dos projetos ou irrelevantes, dependendo da natureza do
aplicativo.]}}

\hypertarget{7jqdvmp9503b}{}\textbf{4.5Licen\c{c}a e Instala\c{c}\~ao}

\textit{\textcolor{blue}{[Os problemas de licen\c{c}a e instala\c{c}\~ao tamb\'em podem impactar diretamente o
esfor\c{c}o de desenvolvimento. Por exemplo, a necessidade de suportar serializa\c{c}\~ao, seguran\c{c}a de senha ou
licen\c{c}a de rede criar\'a requisitos adicionais do sistema que devem ser considerados no esfor\c{c}o de
desenvolvimento.}}

\textit{\textcolor{blue}{Os requisitos de instala\c{c}\~ao tamb\'em podem afetar a codifica\c{c}\~ao ou criar a
necessidade de um software de instala\c{c}\~ao separado.]}}

\hypertarget{iggypmnp2sfk}{}\textbf{5.Recursos do Produto}

\textit{\textcolor{blue}{[Liste e descreva resumidamente os recursos do produto. Recursos s\~ao as capacidades de alto
n\'ivel do sistema que s\~ao necess\'arias para fornecer benef\'icios aos usu\'arios. Cada recurso \'e um servi\c{c}o
desejado externamente que, geralmente, requer uma s\'erie de entradas para alcan\c{c}ar o resultado desejado. Por
exemplo, um recurso de um sistema de rastreamento de problema pode ser a habilidade de fornecer relat\'orios de
tend\^encias. Conforme o modelo de caso de uso toma forma, atualize a descri\c{c}\~ao para se referir aos casos de
uso.}}

\textit{\textcolor{blue}{Como o documento }}\textbf{\textit{\textcolor{blue}{Vis\~ao}}}\textit{\textcolor{blue}{ \'e
revisado por uma ampla variedade de pessoas envolvidas, o n\'ivel de detalhes deve ser geral o suficiente para que
todos entendam. Por\'em, detalhes suficientes devem estar dispon\'iveis para fornecer \`a equipe as informa\c{c}\~oes
necess\'arias para criar um modelo de caso de uso.}}

\textit{\textcolor{blue}{Para gerenciar efetivamente a complexidade do aplicativo, recomenda-se para todo novo sistema,
ou um incremento a um sistema existente, recursos abstra\'idos a um n\'ivel alto o suficiente que resultem 25-99
recursos. Estes recursos fornecem a base fundamental para defini\c{c}\~ao do produto, gerenciamento de escopo e
gerenciamento de projeto. Cada recurso ser\'a expandido em maiores detalhes no modelo de caso de uso.}}

\textit{\textcolor{blue}{Em toda esta se\c{c}\~ao, cada recurso ser\'a externamente observ\'avel por usu\'arios,
operadores ou outros sistemas externos. Estes recursos devem incluir uma descri\c{c}\~ao de funcionalidade e problemas
de utilidade relevantes que devem ser tratados. As seguintes diretrizes se aplicam:}}

\textcolor{blue}{{\textbullet}}\textit{\textcolor{blue}{\ \ Evite o design. Mantenha as descri\c{c}\~oes do recurso em
um n\'ivel geral. Focalize nos recursos necess\'arios e por que (n\~ao como)\ \  eles devem ser implementados.}}

\textcolor{blue}{{\textbullet}}\textit{\textcolor{blue}{\ \ Se voc\^e estiver utilizando o toolkit Rational
RequisitePro, tudo dever\'a ser selecionado como requisitos de tipo para f\'acil refer\^encia e rastreamento.]}}

\hypertarget{m1tc0ej1b5u5}{}\textbf{5.1{\textless}aFeature{\textgreater}}


\bigskip

\hypertarget{jf03z7pnew5o}{}\textbf{5.2{\textless}anotherFeature{\textgreater}}


\bigskip


\bigskip

\hypertarget{j5d69y2r6qxe}{}\textbf{6.Restri\c{c}\~oes}

\textit{\textcolor{blue}{[Observe as restri\c{c}\~oes de design, restri\c{c}\~oes externas ou outras depend\^encias.]}}

\hypertarget{gkp9w1ct89u1}{}\textbf{7.Intervalos de Qualidade}

\textit{\textcolor{blue}{[Defina os intervalos de qualidade quanto ao desempenho, for\c{c}a, toler\^ancia a falhas,
utilidade e caracter\'isticas semelhantes que n\~ao s\~ao capturadas no Conjunto de Recursos.]}}

\hypertarget{h0vdnybw53fh}{}\textbf{8.Preced\^encia e Prioridade}

\textit{\textcolor{blue}{[Defina a prioridade dos diferentes recursos do sistema.]}}

\hypertarget{n9bnzr265250}{}\textbf{9.Outros Requisitos de Produto}

\textit{\textcolor{blue}{[Em um n\'ivel alto, liste padr\~oes aplic\'aveis, requisitos de hardware ou plataforma,
requisitos de desempenho e requisitos ambientais.]}}

\hypertarget{stor88pq7jbs}{}\textbf{9.1Padr\~oes Aplic\'aveis}

\textit{\textcolor{blue}{[Liste todos os padr\~oes com os quais o produto deve estar em conformidade. Estes podem
incluir padr\~oes de comunica\c{c}\~oes legais e reguladores (FDA, UCC) (TCP/IP, ISDN), padr\~oes de conformidade com a
plataforma (Windows, UNIX e assim por diante) e padr\~oes de qualidade e seguran\c{c}a (UL, ISO, CMM).]}}

\hypertarget{pxn3hvmiu1qj}{}\textbf{9.2Requisitos do Sistema}

\textit{\textcolor{blue}{[Defina os requisitos do sistema necess\'arios para suportar o aplicativo. Estes podem incluir
os sistemas operacionais de host suportados e plataformas de rede, configura\c{c}\~oes, mem\'oria, perif\'ericos e
software associado.]}}

\hypertarget{3up1a4xsydnl}{}\textbf{9.3Requisitos de Desempenho}

\textit{\textcolor{blue}{[Utilize esta se\c{c}\~ao para detalhar os requisitos de desempenho. Os problemas de desempenho
podem incluir itens como fatores de carregamento de usu\'ario, capacidade de largura de banda ou comunica\c{c}\~ao,
rendimento do processamento, exatid\~ao e confiabilidade ou tempos de resposta sob uma variedade de condi\c{c}\~oes de
carregamento.]}}

\hypertarget{jfexnf1qeamo}{}\textbf{9.4Requisitos Ambientais}

\textit{\textcolor{blue}{[Detalhe os requisitos ambientais conforme necess\'ario. Para sistemas baseados em hardware, os
problemas ambientais podem incluir temperatura, choque, umidade, radia\c{c}\~ao e assim por diante. Para aplicativos de
software, os fatores ambientais podem incluir condi\c{c}\~oes de uso, ambiente do usu\'ario, disponibilidade de
recursos, problemas de manuten\c{c}\~ao e manipula\c{c}\~ao e recupera\c{c}\~ao de erros.]}}

\hypertarget{3x4l0fj0wcq9}{}\textbf{10.Requisitos de Documenta\c{c}\~ao}

\textit{\textcolor{blue}{[Esta se\c{c}\~ao descreve a documenta\c{c}\~ao que deve ser desenvolvida para suportar a
implementa\c{c}\~ao do aplicativo bem-sucedida.]}}

\hypertarget{evarep984nyh}{}\textbf{10.1Manual do Usu\'ario}

\textit{\textcolor{blue}{[Descreva o objetivo e o conte\'udo do Manual do Usu\'ario. Discuta a extens\~ao desejada, o
n\'ivel de detalhes, a necessidade do \'indice, o gloss\'ario de termos, o tutorial contra a estrat\'egia manual de
refer\^encia e assim por diante. As restri\c{c}\~oes de formata\c{c}\~ao e de impress\~ao tamb\'em devem ser
identificadas.]}}

\hypertarget{v49gfq4f3jh2}{}\textbf{10.2Ajuda On-line}

\textit{\textcolor{blue}{[Muitos aplicativos fornecem um sistema de ajuda on-line para auxiliar o usu\'ario. A natureza
desses sistemas \'e exclusiva para o desenvolvimento do aplicativo uma vez que eles combinam aspectos de
programa\c{c}\~ao (hyperlinks e assim por diante) com aspectos de grava\c{c}\~ao t\'ecnica, como organiza\c{c}\~ao e
apresenta\c{c}\~ao. Muitos descobriram que o desenvolvimento de um sistema de ajuda on-line \'e um projeto dentro de um
projeto que se beneficia do gerenciamento de escopo up-front e da atividade de planejamento.]}}

\hypertarget{z2a13le9bom}{}\textbf{10.3Guias de Instala\c{c}\~ao, Configura\c{c}\~ao e Arquivo LEIA-ME}

\textit{\textcolor{blue}{[Um documento que inclui instru\c{c}\~oes de instala\c{c}\~ao e orienta\c{c}\~oes de
configura\c{c}\~ao \'e importante para uma oferta de solu\c{c}\~ao completa. Al\'em disso, um arquivo LEIA-ME \'e,
geralmente, inclu\'ido como um componente padr\~ao. O arquivo LEIA-ME pode incluir uma se\c{c}\~ao {\textquotedbl}O Que
H\'a de Novo com este Release'' e uma discuss\~ao dos problemas de compatibilidade com releases anteriores. A maioria
dos usu\'arios tamb\'em gosta da documenta\c{c}\~ao que define os erros conhecidos e solu\c{c}\~oes alternativas no
arquivo LEIA-ME.]}}

\hypertarget{xx2mdyy75im6}{}\textbf{10.4Etiquetagem e Empacotamento}

\textit{\textcolor{blue}{[Os aplicativos atualizados fornecem uma apar\^encia e comportamento consistentes que
come\c{c}am com o empacotamento do produto e se manifesta nos menus de instala\c{c}\~ao, telas iniciais, sistemas de
ajuda, di\'alogos de GUI e assim por diante. Esta se\c{c}\~ao define as necessidades e os tipos de etiquetagem a serem
incorporados no c\'odigo. Exemplos incluem observa\c{c}\~oes sobre direitos autorais e patentes, logotipos
corporativos, \'icones padronizados e outros elementos gr\'aficos e assim por diante.]}}

\hypertarget{poj4sk172c2r}{}\textbf{A Atributos de Recursos}

\textit{\textcolor{blue}{[Os recursos recebem atributos que podem ser utilizados para avaliar, rastrear, priorizar e
gerenciar os itens de produtos propostos para implementa\c{c}\~ao. Todos os tipos e atributos de requisitos devem ser
esbo\c{c}ados no Plano de Gerenciamento de Requisitos; por\'em, voc\^e pode listar e descrever resumidamente os
atributos para os recursos escolhidos. As subse\c{c}\~oes a seguir representam um conjunto de atributos de recursos
sugeridos.]}}

\hypertarget{e74foovw8700}{}\textbf{A.1\ \ Status}

\textit{\textcolor{blue}{[Defina ap\'os a negocia\c{c}\~ao e a revis\~ao pela equipe de gerenciamento do projeto.
Controla o andamento durante a defini\c{c}\~ao da linha de base do projeto.]}}


\bigskip


\bigskip

\begin{flushleft}
\tablefirsthead{}
\tablehead{}
\tabletail{}
\tablelasttail{}
\begin{supertabular}{|m{0.9663598in}|m{4.17406in}|}
\hline
~

Propostos

~
 &
~

\ \ \ \ \ \ \ \ \textit{\textcolor{blue}{[Utilizado para descrever recursos que \ \ \ \ \ \ \ \ est\~ao em discuss\~ao,
mas que ainda n\~ao foram revisados e \ \ \ \ \ \ \ \ aceitos pelo {\textquotedbl}canal oficial{\textquotedbl}, como um
grupo de trabalho \ \ \ \ \ \ \ \ que consiste em representantes da equipe do projeto, \ \ \ \ \ \ \ \ gerenciamento de
produtos e comunidade de usu\'arios ou clientes.]}}

~
\\\hline
~

Aprovados

~
 &
~

\ \ \ \ \ \ \ \ \textit{\textcolor{blue}{[Recursos que foram julgados \'uteis e \ \ \ \ \ \ \ \ poss\'iveis e que foram
aprovados para implementa\c{c}\~ao pelo canal \ \ \ \ \ \ \ \ oficial.]}}

~
\\\hline
~

Incorporado

~
 &
~

\ \ \ \ \ \ \ \ \textit{\textcolor{blue}{[Recursos incorporados na linha de base \ \ \ \ \ \ \ \ do produto em um
momento espec\'ifico.]}}

~
\\\hline
\end{supertabular}
\end{flushleft}
\hypertarget{1lw853807j7u}{}\textbf{A.2\ \ Benef\'icio}

\textit{\textcolor{blue}{[Definido pelo Marketing, o gerente de produto ou o analista de neg\'ocio. Todos os requisitos
n\~ao s\~ao criados iguais. Classificar requisitos por seu benef\'icio relativo para o usu\'ario final abre um
di\'alogo com clientes, analistas e membros da equipe de desenvolvimento. Utilizado no gerenciamento do escopo e na
determina\c{c}\~ao da prioridade de desenvolvimento.]}}


\bigskip


\bigskip


\bigskip


\bigskip

\begin{flushleft}
\tablefirsthead{}
\tablehead{}
\tabletail{}
\tablelasttail{}
\begin{supertabular}{|m{0.8205598in}|m{4.33026in}|}
\hline
~

\ \ \ \ \ \ \ \ Cr\'itico

~
 &
~

\ \ \ \ \ \ \ \ \textit{\textcolor{blue}{[Recursos essenciais. Falha na \ \ \ \ \ \ \ \ implementa\c{c}\~ao significa
que o sistema n\~ao atender\'a as \ \ \ \ \ \ \ \ necessidades do cliente. Todos os recursos cr\'iticos devem ser
\ \ \ \ \ \ \ \ implementados no release ou o planejamento falhar\'a.]}}

~
\\\hline
~

\ \ \ \ \ \ \ \ Importante

~
 &
~

\ \ \ \ \ \ \ \ \textit{\textcolor{blue}{[Recursos importantes para a efici\^encia \ \ \ \ \ \ \ \ e a efic\'acia do
sistema para a maioria dos aplicativos. A \ \ \ \ \ \ \ \ funcionalidade n\~ao pode ser facilmente fornecida de outra
\ \ \ \ \ \ \ \ maneira. A falta de inclus\~ao de um recurso importante pode \ \ \ \ \ \ \ \ afetar a satisfa\c{c}\~ao
do cliente ou do usu\'ario ou mesmo a \ \ \ \ \ \ \ \ receita, mas o release n\~ao ser\'a atrasado por causa da falta
de \ \ \ \ \ \ \ \ nenhum recurso importante.]}}

~
\\\hline
~

\'Util

~
 &
~

\ \ \ \ \ \ \ \ \textit{\textcolor{blue}{[Recursos que s\~ao \'uteis em aplicativos \ \ \ \ \ \ \ \ menos t\'ipicos
ser\~ao utilizados com menor freq\"u\^encia ou para \ \ \ \ \ \ \ \ os quais solu\c{c}\~oes alternativas razoavelmente
eficientes podem \ \ \ \ \ \ \ \ ser alcan\c{c}adas. Nenhum impacto significativo de receita ou de
\ \ \ \ \ \ \ \ satisfa\c{c}\~ao do cliente poder\'a ser esperado se tal item n\~ao for \ \ \ \ \ \ \ \ inclu\'ido em
um release.]}}

~
\\\hline
\end{supertabular}
\end{flushleft}

\bigskip

\hypertarget{t98pr3e1edzr}{}\textbf{A.3\ \ Esfor\c{c}o}

\textit{\textcolor{blue}{[Definido pela equipe de desenvolvimento. Como mais recursos requerem mais tempo e recursos do
que outros, estimar o n\'umero de semanas por equipe ou pessoa, linhas de c\'odigo requeridas ou pontos de
fun\c{c}\~ao, por exemplo, \'e a melhor maneira de calcular complexidade e definir expectativas do que pode e n\~ao
pode ser realizado em um determinado quadro de tempo. Utilizado no gerenciamento do escopo e na determina\c{c}\~ao da
prioridade de desenvolvimento.]}}

\hypertarget{qxxknecjlc12}{}\textbf{A.4\ \ Risco}

\textit{\textcolor{blue}{[Definido pela equipe de desenvolvimento com base na probabilidade de que o projeto
experimentar\'a eventos indesej\'aveis, como overruns de custo, atrasos no planejamento ou at\'e mesmo cancelamentos. A
maioria dos gerentes de projeto acha que categorizar os riscos como alto, m\'edio e baixo \'e o suficiente, embora
gradua\c{c}\~oes mais refinadas sejam poss\'iveis. O risco pode freq\"uentemente ser avaliado indiretamente medindo a
variabilidade (intervalo) da estimativa de planejamento da equipe de projetos.]}}

\hypertarget{nxsqrvv9ty9t}{}\textbf{A.5\ \ Estabilidade}

\textit{\textcolor{blue}{[Definida pelo analista e pela equipe de desenvolvimento, baseia-se na probabilidade de que os
recursos ser\~ao alterados ou o entendimento da equipe sobre o recurso ser\'a alterado. Usado para ajudar a estabelecer
as prioridades de desenvolvimento e a determinar os itens que dever\~ao ser extra\'idos.]}}

\hypertarget{dy2rvads3twy}{}\textbf{A.6\ \ Release de Destino}

\textit{\textcolor{blue}{[Registra a vers\~ao do produto pretendida na qual o recurso aparecer\'a primeiro. Este campo
pode ser utilizado para alocar recursos de um documento }}\textbf{\textit{\textcolor{blue}{Vis\~ao
}}}\textit{\textcolor{blue}{em um release de linha de base particular. Quando combinado com o campo de status, sua
equipe pode propor, registrar e discutir v\'arios recursos do release sem confirm\'a-los para o desenvolvimento. Apenas
os recursos cujo Status \'e definido como Incorporado e cujo Release de Destino \'e definido ser\~ao implementados.
Quando ocorre o gerenciamento do escopo, o N\'umero de Vers\~ao do Release de Destino pode ser aumentado de forma que o
item permanecer\'a no documento }}\textbf{\textit{\textcolor{blue}{Vis\~ao}}}\textit{\textcolor{blue}{, mas ser\'a
planejado para um release posterior.]}}

\hypertarget{ijgx8duep97t}{}\textbf{A.7\ \ Designado Para}

\textit{\textcolor{blue}{[Em muitos projetos, os recursos ser\~ao designados a {\textquotedbl}equipes de
recursos{\textquotedbl} respons\'aveis por uma extra\c{c}\~ao maior, gravando os requisitos e a implementa\c{c}\~ao do
software. Esta simples lista suspensa ajudar\'a a todos na equipe do projeto a entender melhor as responsabilidades.]}}

\hypertarget{lmiid6nq4bwf}{}\textbf{A.8\ \ Motivo}

\textit{\textcolor{blue}{[Este campo de texto \'e utilizado para rastrear a origem do recurso solicitado. Os requisitos
existem por motivos espec\'ificos. Este campo registra uma explica\c{c}\~ao ou uma refer\^encia a uma explica\c{c}\~ao.
Por exemplo, a refer\^encia pode ser a uma p\'agina e n\'umero de linha de uma especifica\c{c}\~ao de requisito do
produto ou a um marcador de minuto em um v\'ideo de uma entrevista importante do cliente.]}}


\bigskip


\bigskip


\bigskip
\end{document}
